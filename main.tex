% -*- Mode:TeX -*-

%% IMPORTANT: The official thesis specifications are available at:
%%            http://libraries.mit.edu/archives/thesis-specs/
%%
%%            Please verify your thesis' formatting and copyright
%%            assignment before submission.  If you notice any
%%            discrepancies between these templates and the 
%%            MIT Libraries' specs, please let us know
%%            by e-mailing thesis@mit.edu

%% The documentclass options along with the pagestyle can be used to generate
%% a technical report, a draft copy, or a regular thesis.  You may need to
%% re-specify the pagestyle after you \include  cover.tex.  For more
%% information, see the first few lines of mitthesis.cls. 

%\documentclass[12pt,vi,twoside]{mitthesis}
%%
%%  If you want your thesis copyright to you instead of MIT, use the
%%  ``vi'' option, as above.
%%
%\documentclass[12pt,twoside,leftblank]{mitthesis}
%%
%% If you want blank pages before new chapters to be labelled ``This
%% Page Intentionally Left Blank'', use the ``leftblank'' option, as
%% above. 

\documentclass[12pt,twoside]{mitthesis}
\usepackage{lgrind}
%% These have been added at the request of the MIT Libraries, because
%% some PDF conversions mess up the ligatures.  -LB, 1/22/2014
\usepackage{cmap}
\usepackage[T1]{fontenc}
\pagestyle{plain}

%% This bit allows you to either specify only the files which you wish to
%% process, or `all' to process all files which you \include.
%% Krishna Sethuraman (1990).

\typein [\files]{Enter file names to process, (chap1,chap2 ...), or `all' to
process all files:}
\def\all{all}
\ifx\files\all \typeout{Including all files.} \else \typeout{Including only \files.} \includeonly{\files} \fi

\begin{document}

% -*-latex-*-
%
% For questions, comments, concerns or complaints:
% thesis@mit.edu
%
%
% $Log: cover.tex,v $
% Revision 1.8  2008/05/13 15:02:15  jdreed
% Degree month is June, not May.  Added note about prevdegrees.
% Arthur Smith's title updated
%
% Revision 1.7  2001/02/08 18:53:16  boojum
% changed some \newpages to \cleardoublepages
%
% Revision 1.6  1999/10/21 14:49:31  boojum
% changed comment referring to documentstyle
%
% Revision 1.5  1999/10/21 14:39:04  boojum
% *** empty log message ***
%
% Revision 1.4  1997/04/18  17:54:10  othomas
% added page numbers on abstract and cover, and made 1 abstract
% page the default rather than 2.  (anne hunter tells me this
% is the new institute standard.)
%
% Revision 1.4  1997/04/18  17:54:10  othomas
% added page numbers on abstract and cover, and made 1 abstract
% page the default rather than 2.  (anne hunter tells me this
% is the new institute standard.)
%
% Revision 1.3  93/05/17  17:06:29  starflt
% Added acknowledgements section (suggested by tompalka)
%
% Revision 1.2  92/04/22  13:13:13  epeisach
% Fixes for 1991 course 6 requirements
% Phrase "and to grant others the right to do so" has been added to
% permission clause
% Second copy of abstract is not counted as separate pages so numbering works
% out
%
% Revision 1.1  92/04/22  13:08:20  epeisach

% NOTE:
% These templates make an effort to conform to the MIT Thesis specifications,
% however the specifications can change.  We recommend that you verify the
% layout of your title page with your thesis advisor and/or the MIT
% Libraries before printing your final copy.
\title{Performance Optimization of a Split-Value Voting System}

\author{Charles Z. Liu}
% If you wish to list your previous degrees on the cover page, use the
% previous degrees command:
%       \prevdegrees{A.A., Harvard University (1985)}
% You can use the \\ command to list multiple previous degrees
%       \prevdegrees{B.S., University of California (1978) \\
%                    S.M., Massachusetts Institute of Technology (1981)}
%\prevdegrees{S.B., Massachusetts Institute of Technology (2014)}

\department{Department of Electrical Engineering and Computer Science}

% If the thesis is for two degrees simultaneously, list them both
% separated by \and like this:
% \degree{Doctor of Philosophy \and Master of Science}
\degree{Master of Engineering in Electrical Engineering and Computer Science}

% As of the 2007-08 academic year, valid degree months are September,
% February, or June.  The default is June.
\degreemonth{June}
\degreeyear{2014}
\thesisdate{June 8, 2015}

%% By default, the thesis will be copyrighted to MIT.  If you need to copyright
%% the thesis to yourself, just specify the `vi' documentclass option.  If for
%% some reason you want to exactly specify the copyright notice text, you can
%% use the \copyrightnoticetext command.
%\copyrightnoticetext{\copyright IBM, 1990.  Do not open till Xmas.}

% If there is more than one supervisor, use the \supervisor command
% once for each.
\supervisor{Prof. Ronald L. Rivest}{}

% This is the department committee chairman, not the thesis committee
% chairman.  You should replace this with your Department's Committee
% Chairman.
\chairman{Prof. Albert R. Meyer}{Chairman, Masters of Engineering Thesis Committee}

% Make the titlepage based on the above information.  If you need
% something special and can't use the standard form, you can specify
% the exact text of the titlepage yourself.  Put it in a titlepage
% environment and leave blank lines where you want vertical space.
% The spaces will be adjusted to fill the entire page.  The dotted
% lines for the signatures are made with the \signature command.
\maketitle

% The abstractpage environment sets up everything on the page except
% the text itself.  The title and other header material are put at the
% top of the page, and the supervisors are listed at the bottom.  A
% new page is begun both before and after.  Of course, an abstract may
% be more than one page itself.  If you need more control over the
% format of the page, you can use the abstract environment, which puts
% the word "Abstract" at the beginning and single spaces its text.

%% You can either \input (*not* \include) your abstract file, or you can put
%% the text of the abstract directly between the \begin{abstractpage} and
%% \end{abstractpage} commands.

% First copy: start a new page, and save the page number.
\cleardoublepage
% Uncomment the next line if you do NOT want a page number on your
% abstract and acknowledgments pages.
% \pagestyle{empty}
\setcounter{savepage}{\thepage}
\begin{abstractpage}
% $Log: abstract.tex,v $
% Revision 1.1  93/05/14  14:56:25  starflt
% Initial revision
%
% Revision 1.1  90/05/04  10:41:01  lwvanels
% Initial revision
%
%
%% The text of your abstract and nothing else (other than comments) goes here.
%% It will be single-spaced and the rest of the text that is supposed to go on
%% the abstract page will be generated by the abstractpage environment.  This
%% file should be \input (not \include 'd) from cover.tex.

As digital technology becomes more powerful and commonplace, the benefits of using computers to conduct elections become more apparent. In today's elections dominated by paper ballots, we cannot be certain that the election result is correct. Many parts of the world are plagued by corrupt election officials and rampant bribery.

In this thesis, we review the cryptographic techniques available for designing a secure election system and introduce a system designed around a verifiable mixnet using split-value commitments. The main work done in this thesis is a series of performance optimizations to the existing prototype, greatly improving the real-world viability of the system. Finally, we suggest further work that can be done to improve performance, fault tolerance, and security.

\end{abstractpage}

% Additional copy: start a new page, and reset the page number.  This way,
% the second copy of the abstract is not counted as separate pages.
% Uncomment the next 6 lines if you need two copies of the abstract
% page.
% \setcounter{page}{\thesavepage}
% \begin{abstractpage}
% % $Log: abstract.tex,v $
% Revision 1.1  93/05/14  14:56:25  starflt
% Initial revision
%
% Revision 1.1  90/05/04  10:41:01  lwvanels
% Initial revision
%
%
%% The text of your abstract and nothing else (other than comments) goes here.
%% It will be single-spaced and the rest of the text that is supposed to go on
%% the abstract page will be generated by the abstractpage environment.  This
%% file should be \input (not \include 'd) from cover.tex.

As digital technology becomes more powerful and commonplace, the benefits of using computers to conduct elections become more apparent. In today's elections dominated by paper ballots, we cannot be certain that the election result is correct. Many parts of the world are plagued by corrupt election officials and rampant bribery.

In this thesis, we review the cryptographic techniques available for designing a secure election system and introduce a system designed around a verifiable mixnet using split-value commitments. The main work done in this thesis is a series of performance optimizations to the existing prototype, greatly improving the real-world viability of the system. Finally, we suggest further work that can be done to improve performance, fault tolerance, and security.

% \end{abstractpage}

\cleardoublepage

\section*{Acknowledgments}

This is the acknowledgements section.  You should replace this with your
own acknowledgements.

%%%%%%%%%%%%%%%%%%%%%%%%%%%%%%%%%%%%%%%%%%%%%%%%%%%%%%%%%%%%%%%%%%%%%%
% -*-latex-*-

% Some departments (e.g. 5) require an additional signature page.  See
% signature.tex for more information and uncomment the following line if
% applicable.
% % -*- Mode:TeX -*-
%
% Some departments (e.g. Chemistry) require an additional cover page
% with signatures of the thesis committee.  Please check with your
% thesis advisor or other appropriate person to determine if such a 
% page is required for your thesis.  
%
% If you choose not to use the "titlepage" environment, a \newpage
% commands, and several \vspace{\fill} commands may be necessary to
% achieve the required spacing.  The \signature command is defined in
% the "mitthesis" class
%
% The following sample appears courtesy of Ben Kaduk <kaduk@mit.edu> and
% was used in his June 2012 doctoral thesis in Chemistry. 

\begin{titlepage}
\begin{large}
This doctoral thesis has been examined by a Committee of the Department
of Chemistry as follows:

\signature{Professor Jianshu Cao}{Chairman, Thesis Committee \\
   Professor of Chemistry}

\signature{Professor Troy Van Voorhis}{Thesis Supervisor \\
   Associate Professor of Chemistry}

\signature{Professor Robert W. Field}{Member, Thesis Committee \\
   Haslam and Dewey Professor of Chemistry}
\end{large}
\end{titlepage}


\pagestyle{plain}
  % -*- Mode:TeX -*-
%% This file simply contains the commands that actually generate the table of
%% contents and lists of figures and tables.  You can omit any or all of
%% these files by simply taking out the appropriate command.  For more
%% information on these files, see appendix C.3.3 of the LaTeX manual.
\tableofcontents
\newpage
\listoffigures
\newpage
\listoftables


%% This is an example first chapter.  You should put chapter/appendix that you
%% write into a separate file, and add a line \include{yourfilename} to
%% main.tex, where `yourfilename.tex' is the name of the chapter/appendix file.
%% You can process specific files by typing their names in at the
%% \files=
%% prompt when you run the file main.tex through LaTeX.
\chapter{Introduction} \label{intro}

As a cornerstone of democratic governments, elections are the subject of intense scrutiny. Voters expect that the systems and procedures put in place are easy to follow, efficient, and inexpensive. However, history tells us that in practice, this may not always be the case.

Perhaps the most well-known example of the failure of voting systems in the current generation is the 2000 Presidential Election in the United States, most notably the series of recounts in Florida that delayed the nationwide election result by several weeks and opened cases in the Florida Supreme Court and the United States Supreme Court. Even in the aftermath of the election, studies were inconclusive as to whether the county-specific recounts or a statewide recount would have swung the results in Al Gore's favor. The problem was exacerbated by the media's premature declaration of the election outcome, George W. Bush's narrow margin of victory (around 500 votes), and controversy surrounding the ``butterfly ballots'' that were used in Palm Beach for the first time \cite{wiki:2000-fl-pres}.

Studies conducted after the 2000 Election in Florida reveal many issues that appear to be quite subtle in the larger picture of a Presidential election: the public's familiarity with a new ballot layout and instructions, the timing of media releases and exit polls, and even the degree to which paper ballots are cleanly marked or punctured by the voter. In the aftermath of the chaos, Congress passed the Help America Vote Act, which attempted to help states upgrade their election infrastructure to prevent similar issues four years later. Many states turned to electronic voting systems. Unfortunately, these systems have been met with criticisms over their accessibility, relability, and security. Furthermore, many of these proprietary systems have reached or are approaching the ends of their lifetimes, prompting another wave of research and development on secure, verifiable, efficient, and reliable electronic voting systems.

Elections are not only thwarted by the failures of the mechanisms used in them. Reports of manipulated or lost ballots, coerced voters, and simply corrupt election officials are no surprise in certain parts of the world even today. In decades past, voters in Italian elections often had the option of selecting ranked preferences among a long list of candidates. The clever ``Italian attack'' \cite{naish:italian-attack} reportedly used by the Mafia involved coercing a voter to vote for the coercer's first choice candidate, followed by a sequence of low-probability candidates in some predetermined order. With a long list of candidates, the chance that another ballot contains the exact same sequence in the same slots is very low; thus, if the votes are publicly broadcast following the election, the sequence becomes a signature for the coerced voter that the coercer can check, even if all other identifiable information is stripped.

Clearly, there is much more to be done in advancing election systems. In this section, we briefly survey the history of voting systems, including electronic voting systems, and conclude with a discussion of desirable properties in electronic voting systems.

\section{A Brief History of Voting Systems} \label{intro:history}

The first recorded elections occurred in the sixth century BC in ancient Greece, where adult males were charged with the responsibility to vote at meetings of the assembly. In the beginning, voting was typically done by a raise of hands in a public space. This developed into the use of colored stones, where voters would select a colored stone out of two possible colors and place it into a central jar. Although we use secret ballots today without a second thought, the first ``secret ballots'' were used in the ancient Greek and Roman civilizations, where records exist of contraptions that allowed a voter to hide the stone as it was deposited into the jar. By the end of the nineteenth century, most Western countries, including the United States, Australia, and France, demanded the use of a secret ballot at elections \cite{wiki:secret-ballot}, and voter privacy was a non-negotiable property of official elections.

As populations grew and election officials looked for ways to make elections more efficient, mechanized voting machines were developed. Lever machines were introduced in New York during the 1892 Election. Punch card systems were introduced in 1965 and were also widely used until they drew sharp criticism during the 2000 Election in Florida. Lever machines, punch cards, and other mechanical and electrical systems such as optical scanners provided different tradeoffs in cost, convenience, efficiency, usability, accessibility, and security. However, in the past decade, growth in both the power and popularity of digital technology has enabled a shift towards electronic voting systems.

\section{The Rise of Electronic Voting} \label{intro:evoting}

In the aftermath of the 2000 Election, the Help America Vote Act instituted a series of requirements on the states which aimed to replace punch card and lever machines, establish standards on voter privacy and accessibility for voters with disabilities, and improve the usability of voting systems. The number of direct-recording electronic (DRE) voting machines grew in response. DREs have a number of advantages over their mechanical counterparts.
\begin{itemize}
\item They enable rapid, accurate tallying of votes.
\item They may include numerous accessibility functions, notably those for the visually impaired.
\item They may include many languages.
\item They may catch errors such as undervoting and overvoting before the vote is cast, preventing invalid ballots from being discarded.
\end{itemize}

However, DREs are often complex systems containing proprietary hardware and software, making it difficult to verify their correctness and security properties; indeed, studies have shown the existence of numerous security vulnerabilities in many commercially available electronic voting systems. Because of the lack of a paper trail, any errors in the system are unrecoverable.

A solution to the lack of verifiability in DREs was proposed in 1992 by Mercuri \cite{mercuri:vvpat}, creating a voter-verified paper audit trail (VVPAT) system. In the simplest VVPAT system, a DRE machine is augmented with a printer. In addition to submitting the vote electronically, the machine prints out a human-readable copy of the voter's selections. The voter may then confirm that the paper record is indeed correct before depositing it into a ballot box. In case of a disputed election, recount, or malfunctioning machines, the paper records may be consuled. VVPAT systems may be extended so that the voters may keep a printed receipt; however, care must be taken to ensure that the printed receipt does not reveal the voter's vote to anyone else.

Today, VVPAT systems are the most common type of electronic voting system; 27 states in the United States require a paper audit trail by law and an additional 18 states use them in their elections.

\section{Desirable Properties in Electronic Voting Systems} \label{intro:props}

The two main propeties desired in an election are \emph{privacy} and \emph{verifiability}.

\emph{Privacy} concerns the ability of an adversary to gain information on a voter's choices in an election. A lack of privacy results in the possibility of coercion or bribery as seen in the earlier example of the ``Italian attack''. Note that the adversary is not limited to an outside third party; privacy is broken if a corrupt election official may associate any vote with the corresponding voter. As voting systems have become more complex and electronic voting systems have become more popular, the notion of privacy has evolved.

The simplest form of privacy is \emph{ballot privacy}, which stipulates that the contents of the ballot must remain secret. With typical paper ballots such as those used today in the United States, ballot privacy guarantees that there can be no association made between a voter and her ballot, thereby providing privacy to the voter. However, with electronic voting systems, a stronger definition of privacy is desired. \emph{Receipt freeness} or \emph{incoercibility}, first introduced in \cite{bt94}, concerns the inability of the voter to later prove how she voted. Equivalently, it is impossible to effectively coerce a voter by forcing her to reveal her vote following the election. The concept of receipt freeness grew in importance as electronic voting systems often leave a receipt that the voter may bring out of the polling site.

In certain areas of the United States and other countries including Switzerland and Estonia, Internet voting has become commonplace. In Estonia, for example, voters may scan their government issued ID cards to authenticate themselves into a voting website. Many more states and countries have vote-by-mail, in which ballots may be mailed ahead of the election date. A stronger notion of \emph{coercion resistance} due to the work of \cite{jcj05} identified the need to prevent adversaries from jeopardizing voter privacy in these cases, for example, by stealing or forcing a voter to reveal her authentication credentials.

\emph{Verifiability} concerns whether individual voters or the public as a whole may verify that the election outcome faithfully represents the selections of the voters. In particular, \emph{individual verifiability} refers to the ability of a voter to verify that her vote was recorded in the election. \emph{Universal verifiability} refers to the ability of anyone in the public to verify that the declared election outcome matches the set of votes that were recorded in the election.

Although today's paper ballots provide good privacy, they do so at immense cost to verifiability. Despite thorough documentation of best practices concerning how ballots should be handled, who should touch them, how they should be sealed, etc. \cite{ace-electoral-network}, it remains practically impossible for a voter to know for certain that her ballot was not lost in transit. Since ballots are not released to the public, voters must trust that the outcome declared by election officials represents the accurate tallying of all ballots. As seen in the 2000 Presidential Election example, good verifiability is not a strength in the systems used in the majority of the United States.

For electronic voting systems, however, we may achieve verifiability while preserving privacy by using cryptography where appropriate. For example, a vote may be encrypted before being cast; the encrypted votes may then be added homomorphically to yield an encryption of the final election outcome, enabling the election officials to compute the result of an election without revealing any individual votes. Threshold encryption may strengthen this. For example, by requiring that 3 out of 5 election officials be present to obtain the decryption key, we reduce the chance of a malicious election official revealing individual votes.

By introducing a secure, publicly viewable bulletin board, we may achieve verifiability at every stage from when the vote is cast until the election result is revealed.

\begin{definition} An \emph{end-to-end verifiable} voting system is one that satisfies the following properties.
\begin{itemize}
\item \emph{Cast-as-intended}. It may be verified that the encryption or other representation of the vote is indeed a representation of the proper vote. Methods to ensure cast-as-intended behavior include Chaum's work \cite{chaum04} or the challenge/response method proposed by Neff \cite{neff04} and Benaloh \cite{benaloh06}.
\item \emph{Recorded-as-cast}. It may be verified by the voter that her vote is properly taken into account. For example, voters may be given identifiers, and a public bulletin board may contain a listing of voter identifiers and corresponding encrypted votes. We must ensure that this step does not reveal the plaintext vote to an adversary.
\item \emph{Tallied-as-recorded}. It may be verified by anyone that the final election result is correct given the individual votes. For example, a list of encrypted votes would be homomorphically added, and then the election officials would provide a zero-knowledge proof that the election result corresponds to a decryption of the tallied votes without revealing the decryption key.
\end{itemize}
\end{definition}

\section{Conclusion} \label{intro:conclusion}

In this chapter, we presented an overview of voting systems: their history, implementations, and downfalls. We discussed the various notions of privacy and verifiability and introduced the concept of an end-to-end verifiable voting system. In the remainder of this thesis, we survey the relevant technologies used in end-to-end verifiable voting systems in Chapter 2, introduce the split-value system and its implementation in Chapter 3, present performance optimizations to the system in Chapter 4, discuss opportunities for future work in Chapter 5, and conclude in Chapter 6.

\chapter{Electronic Voting Systems and Technologies}

TODO: Survey the other electronic voting systems and the crypto used in them, such as helios, votebox, starvote, wombat.

TODO: Survey mixnets for design and performance.

\appendix
\chapter{Tables}

\begin{table}
\caption{Armadillos}
\label{arm:table}
\begin{center}
\begin{tabular}{||l|l||}\hline
Armadillos & are \\\hline
our	   & friends \\\hline
\end{tabular}
\end{center}
\end{table}

\clearpage
\newpage

\chapter{Figures}

\vspace*{-3in}

\begin{figure}
\vspace{2.4in}
\caption{Armadillo slaying lawyer.}
\label{arm:fig1}
\end{figure}
\clearpage
\newpage

\begin{figure}
\vspace{2.4in}
\caption{Armadillo eradicating national debt.}
\label{arm:fig2}
\end{figure}
\clearpage
\newpage

%% This defines the bibliography file (main.bib) and the bibliography style.
%% If you want to create a bibliography file by hand, change the contents of
%% this file to a `thebibliography' environment.  For more information 
%% see section 4.3 of the LaTeX manual.
\begin{singlespace}
\bibliography{main}
\bibliographystyle{plain}
\end{singlespace}

\end{document}

