%% This is an example first chapter.  You should put chapter/appendix that you
%% write into a separate file, and add a line \include{yourfilename} to
%% main.tex, where `yourfilename.tex' is the name of the chapter/appendix file.
%% You can process specific files by typing their names in at the
%% \files=
%% prompt when you run the file main.tex through LaTeX.
\chapter{Introduction} \label{intro}

As a cornerstone of democratic governments, elections are the subject of intense scrutiny. Voters expect that the systems and procedures put in place are easy to follow, efficient, and inexpensive. However, history tells us that in practice, this may not always be the case.

Perhaps the most well-known example of the failure of voting systems in the current generation is the 2000 Presidential Election in the United States, most notably the series of recounts in Florida that delayed the nationwide election result by several weeks and opened cases in the Florida Supreme Court and the United States Supreme Court. Even in the aftermath of the election, studies were inconclusive as to whether the county-specific recounts or a statewide recount would have swung the results in Al Gore's favor. The problem was exacerbated by the media's premature declaration of the election outcome, George W. Bush's narrow margin of victory (around 500 votes), and controversy surrounding the ``butterfly ballots'' that were used in Palm Beach for the first time \cite{wiki:2000-fl-pres}.

Studies conducted after the 2000 Election in Florida reveal many issues that appear to be quite subtle in the larger picture of a Presidential election: the public's familiarity with a new ballot layout and instructions, the timing of media releases and exit polls, and even the degree to which paper ballots are cleanly marked or punctured by the voter. In the aftermath of the chaos, Congress passed the Help America Vote Act, which attempted to help states upgrade their election infrastructure to prevent similar issues four years later. Many states turned to electronic voting systems. Unfortunately, these systems have been met with criticisms over their accessibility, relability, and security. Furthermore, many of these proprietary systems have reached or are approaching the ends of their lifetimes, prompting another wave of research and development on secure, verifiable, efficient, and reliable electronic voting systems.

Elections are not only thwarted by the failures of the mechanisms used in them. Reports of manipulated or lost ballots, coerced voters, and simply corrupt election officials are no surprise in certain parts of the world even today. In decades past, voters in Italian elections often had the option of selecting ranked preferences among a long list of candidates. The clever ``Italian attack'' \cite{naish:italian-attack} reportedly used by the Mafia involved coercing a voter to vote for the coercer's first choice candidate, followed by a sequence of low-probability candidates in some predetermined order. With a long list of candidates, the chance that another ballot contains the exact same sequence in the same slots is very low; thus, if the votes are publicly broadcast following the election, the sequence becomes a signature for the coerced voter that the coercer can check, even if all other identifiable information is stripped.

Clearly, there is much more to be done in advancing election systems. In this section, we briefly survey the history of voting systems, including electronic voting systems, and conclude with a discussion of desirable properties in electronic voting systems.

\section{A Brief History of Voting Systems} \label{intro:history}

The first recorded elections occurred in the sixth century BC in ancient Greece, where adult males were charged with the responsibility to vote at meetings of the assembly. In the beginning, voting was typically done by a raise of hands in a public space. This developed into the use of colored stones, where voters would select a colored stone out of two possible colors and place it into a central jar. Although we use secret ballots today without a second thought, the first ``secret ballots'' were used in the ancient Greek and Roman civilizations, where records exist of contraptions that allowed a voter to hide the stone as it was deposited into the jar. By the end of the nineteenth century, most Western countries, including the United States, Australia, and France, demanded the use of a secret ballot at elections \cite{wiki:secret-ballot}, and voter privacy was a non-negotiable property of official elections.

As populations grew and election officials looked for ways to make elections more efficient, mechanized voting machines were developed. Lever machines were introduced in New York during the 1892 Election. Punch card systems were introduced in 1965 and were also widely used until they drew sharp criticism during the 2000 Election in Florida. Lever machines, punch cards, and other mechanical and electrical systems such as optical scanners provided different tradeoffs in cost, convenience, efficiency, usability, accessibility, and security. However, in the past decade, growth in both the power and popularity of digital technology has enabled a shift towards electronic voting systems.

\section{The Rise of Electronic Voting} \label{intro:evoting}

In the aftermath of the 2000 Election, the Help America Vote Act instituted a series of requirements on the states which aimed to replace punch card and lever machines, establish standards on voter privacy and accessibility for voters with disabilities, and improve the usability of voting systems. The number of direct-recording electronic (DRE) voting machines grew in response. DREs have a number of advantages over their mechanical counterparts.
\begin{itemize}
\item They enable rapid, accurate tallying of votes.
\item They may include numerous accessibility functions, notably those for the visually impaired.
\item They may include many languages.
\item They may catch errors such as undervoting and overvoting before the vote is cast, preventing invalid ballots from being discarded.
\end{itemize}

However, DREs are often complex systems containing proprietary hardware and software, making it difficult to verify their correctness and security properties; indeed, studies have shown the existence of numerous security vulnerabilities in many commercially available electronic voting systems. Because of the lack of a paper trail, any errors in the system are unrecoverable.

A solution to the lack of verifiability in DREs was proposed in 1992 by Mercuri \cite{mercuri:vvpat}, creating a voter-verified paper audit trail (VVPAT) system. In the simplest VVPAT system, a DRE machine is augmented with a printer. In addition to submitting the vote electronically, the machine prints out a human-readable copy of the voter's selections. The voter may then confirm that the paper record is indeed correct before depositing it into a ballot box. In case of a disputed election, recount, or malfunctioning machines, the paper records may be consuled. VVPAT systems may be extended so that the voters may keep a printed receipt; however, care must be taken to ensure that the printed receipt does not reveal the voter's vote to anyone else.

Today, VVPAT systems are the most common type of electronic voting system; 27 states in the United States require a paper audit trail by law and an additional 18 states use them in their elections.

\section{Desirable Properties in Electronic Voting Systems} \label{intro:props}

The two main propeties desired in an election are \emph{privacy} and \emph{verifiability}.

\emph{Privacy} concerns the ability of an adversary to gain information on a voter's choices in an election. A lack of privacy results in the possibility of coercion or bribery as seen in the earlier example of the ``Italian attack''. Note that the adversary is not limited to an outside third party; privacy is broken if a corrupt election official may associate any vote with the corresponding voter. As voting systems have become more complex and electronic voting systems have become more popular, the notion of privacy has evolved.

The simplest form of privacy is \emph{ballot privacy}, which stipulates that the contents of the ballot must remain secret. With typical paper ballots such as those used today in the United States, ballot privacy guarantees that there can be no association made between a voter and her ballot, thereby providing privacy to the voter. However, with electronic voting systems, a stronger definition of privacy is desired. \emph{Receipt freeness} or \emph{incoercibility}, first introduced in \cite{bt94}, concerns the inability of the voter to later prove how she voted. Equivalently, it is impossible to effectively coerce a voter by forcing her to reveal her vote following the election. The concept of receipt freeness grew in importance as electronic voting systems often leave a receipt that the voter may bring out of the polling site.

In certain areas of the United States and other countries including Switzerland and Estonia, Internet voting has become commonplace. In Estonia, for example, voters may scan their government issued ID cards to authenticate themselves into a voting website. Many more states and countries have vote-by-mail, in which ballots may be mailed ahead of the election date. A stronger notion of \emph{coercion resistance} due to the work of \cite{jcj05} identified the need to prevent adversaries from jeopardizing voter privacy in these cases, for example, by stealing or forcing a voter to reveal her authentication credentials.

\emph{Verifiability} concerns whether individual voters or the public as a whole may verify that the election outcome faithfully represents the selections of the voters. In particular, \emph{individual verifiability} refers to the ability of a voter to verify that her vote was recorded in the election. \emph{Universal verifiability} refers to the ability of anyone in the public to verify that the declared election outcome matches the set of votes that were recorded in the election.

Although today's paper ballots provide good privacy, they do so at immense cost to verifiability. Despite thorough documentation of best practices concerning how ballots should be handled, who should touch them, how they should be sealed, etc. \cite{ace-electoral-network}, it remains practically impossible for a voter to know for certain that her ballot was not lost in transit. Since ballots are not released to the public, voters must trust that the outcome declared by election officials represents the accurate tallying of all ballots. As seen in the 2000 Presidential Election example, good verifiability is not a strength in the systems used in the majority of the United States.

For electronic voting systems, however, we may achieve verifiability while preserving privacy by using cryptography where appropriate. For example, a vote may be encrypted before being cast; the encrypted votes may then be added homomorphically to yield an encryption of the final election outcome, enabling the election officials to compute the result of an election without revealing any individual votes. Threshold encryption may strengthen this. For example, by requiring that 3 out of 5 election officials be present to obtain the decryption key, we reduce the chance of a malicious election official revealing individual votes.

By introducing a secure, publicly viewable bulletin board, we may achieve verifiability at every stage from when the vote is cast until the election result is revealed.

\begin{definition} An \emph{end-to-end verifiable} voting system is one that satisfies the following properties.
\begin{itemize}
\item \emph{Cast-as-intended}. It may be verified that the encryption or other representation of the vote is indeed a representation of the proper vote. Methods to ensure cast-as-intended behavior include Chaum's work \cite{chaum04} or the challenge/response method proposed by Neff \cite{neff04} and Benaloh \cite{benaloh06}.
\item \emph{Recorded-as-cast}. It may be verified by the voter that her vote is properly taken into account. For example, voters may be given identifiers, and a public bulletin board may contain a listing of voter identifiers and corresponding encrypted votes. We must ensure that this step does not reveal the plaintext vote to an adversary.
\item \emph{Tallied-as-recorded}. It may be verified by anyone that the final election result is correct given the individual votes. For example, a list of encrypted votes would be homomorphically added, and then the election officials would provide a zero-knowledge proof that the election result corresponds to a decryption of the tallied votes without revealing the decryption key.
\end{itemize}
\end{definition}

\section{Conclusion} \label{intro:conclusion}

In this chapter, we presented an overview of voting systems: their history, implementations, and downfalls. We discussed the various notions of privacy and verifiability and introduced the concept of an end-to-end verifiable voting system. In the remainder of this thesis, we survey the relevant technologies used in end-to-end verifiable voting systems in Chapter 2, introduce the split-value system and its implementation in Chapter 3, present performance optimizations to the system in Chapter 4, discuss opportunities for future work in Chapter 5, and conclude in Chapter 6.
